\chapter{总结与展望}\label{ch:con}

\section{总结}

在SLAM算法中,状态估计算法的速度和精度一定程度上决定了SLAM应用的可用性。传统的批量式集束调整求解器为了保证通用性,牺牲了一定的效率,而现有的基于增量式集束调整的求解器虽然效率很高,但由于求解方法固定,通常只适用于特定类型的问题,且由于耦合程度过高,难以在其他SLAM系统中使用。本文基于以上分析,提出了面向VISLAM的基于增量式集束调整的通用求解框架,并针对现有的基于增量式舒尔补的集束调整算法中的问题,提出了因子图编码的增量式舒尔补、HLMDL、PBT回代等算法,在保证一定精度和变量一致性的前提下提升了求解效率。

另一方面,本文提出的框架在非线性优化部分,提供了VISLAM中常用的基于IMU预积分的目标函数接口和重投影误差目标函数,同时保留了可扩展性,用户可以直接使用内置的目标函数,也可以自定义目标函数;在线性求解部分,基于块状稀疏矩阵,提供了Cholesky分解法、I-PCG等常用的线性求解接口,同时用户也可以定制专用的线性求解器。最后,在模拟数据和SLAM++提供的集束调整数据集上进行了测试,通过实验验证了求解器的效率和精度。

\section{展望}

增量式集束调整利用了SLAM问题的稀疏性和局部性,极大地提升了效率。但是仍有一些问题可以改进:
\begin{itemize}
    \item 在发生回路闭合的时候,通常大量的状态变量都需要进行更新,增量式集束调整会退化到普通的批量式集束调整方法,效率严重下降;
    \item 状态边缘化通常会影响问题的稀疏性,频繁的边缘化操作则会导致系统变得十分稠密,不利于基于块状稀疏矩阵的求解算法。可以考虑一些稀疏化方法,使用一些近似的较为稀疏的先验约束来代替原本由边缘化得到的较为稠密的先验约束,以加快计算。
    \item 本文实现的增量式集束调整框架仍然采用了非增量的Cholesky分解或I-PCG迭代来求解舒尔补方程。如何在保证数值稳定性的情况下进一步利用增量特点,加速线性舒尔补方程的求解(如使用增量式Cholesky分解),是本文的下一步工作。
    \item 集束调整算法的线性化部分、舒尔补部分仍有可能采用更高效的并行算法提升效率。
\end{itemize}
如何在集束调整求解器中解决以上问题,或是在遇到以上问题的时候保持计算效率和精度,仍是一个很大的挑战。
