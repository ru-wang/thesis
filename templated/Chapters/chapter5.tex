\chapter{总结与展望}\label{ch:con}

\section{总结}

在SLAM算法中,状态估计算法的速度和精度一定程度上决定了SLAM应用的可用性。传统的批量式集束优化求解器为了保证通用性,牺牲了一定的效率,而现有的基于增量式集束优化的求解器虽然效率很高,但由于耦合程度过高,难以在其他SLAM系统中使用。本文基于以上分析,提出了基于增量式集束优化的通用求解器,在保证效率和精度的基础上提供了很好的通用性,可以很容易地集成到其他SLAM系统中。

在非线性优化部分,提供了IMU预积分的目标函数接口和重投影误差目标函数接口,同时保留扩展性,用户可以直接使用内置的目标函数,也可以自定义目标函数;在线性部分,使用了增量舒尔补的方法大幅提升了线性化过程的效率;并基于块状稀疏矩阵,提供了Cholesky分解法、预处理共轭梯度法等常用的线性求解接口,同时用户也可以定制专用的线性求解器。最后,在模拟数据和SLAM++提供的集束优化数据集上进行了测试,通过实验验证了求解器的效率和精度。

\section{展望}

增量式集束优化利用了SLAM问题的稀疏性和局部性,极大地提升了效率。但是仍有一些情况会降低增量式集束优化的效率:
\begin{enumerate}
    \item 在发生回路闭合的时候,通常大量的状态变量都需要进行更新,增量式集束优化会退化到普通的批量式集束优化方法,效率严重下降;
    \item 状态边缘化通常会影响问题的稀疏性,频繁的边缘化操作则会导致系统变得十分稠密,不利于基于块状稀疏矩阵的求解算法。
\end{enumerate}
如何在集束优化过程中避免以上问题,或是在遇到以上问题的时候保持计算效率和精度,仍是一个很大的挑战。
