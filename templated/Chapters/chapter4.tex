\chapter{实验和结论}\label{ch:exp}

本章将通过模拟数据和真实数据,测试本文提出的增量式集束调整算法和传统批量式集束调整算法的速度和精度。Ceres-Solver是Google为了求解SfM问题而推出的非线性最小二乘求解器,经过多年的发展,Ceres-Solver已经广泛地应用在多个SLAM系统中,包括OKVIS、VINS-Mono等,并已成为开源社区中最为流行的通用非线性最小二乘求解器之一。作为一个通用的求解器,Ceres-Solver提供了稠密版本和稀疏版本的多种批量式数值优化算法实现(包括LM方法、DL法等),具有极高的数值稳定性和较高的效率。

\section{测试设置}

测试数据包含模拟生成的相机-IMU-三维点的数据集Cubic和包含相机-三维点的VSLAM真实数据集Cathedral\citep{kim2014influence}和Venice\citep{kummerle2011g}。其中Cubic数据集为模拟生成的带高斯噪声的VISLAM数据集,包含IMU预积分的结果和协方差、相机状态的初值、视觉观测。Cathedral和Venice为真实数据集,包含相机状态的初值、三维点状态的初值和视觉观测。这样的设置省去了SLAM前端视觉匹配、IMU积分的过程,保证所有算法求解的问题相同。几组数据具体包含的内容如表~\ref{tab:dataset}~所示,由于Ceres-Solver求解完整的Venice数据集失败,顾只取了该数据集前$10$个相机状态,大约包含$8000$个三维点。

{
\linespread{1}
\begin{table}[htb!]
\zihao{5}
\caption{测试数据集明细}
\label{tab:dataset}
\centering
\begin{tabular}{l|cccc}
    \toprule
    数据集    & 相机数量 & 三维点数量 & IMU目标函数 & 重投影误差目标函数 \\ \midrule
    Cubic     &       50 &        163 &          49 &                789 \\
    Venice    &       11 &       8366 &           0 &              20271 \\
    Cathedral &       92 &      57957 &           0 &             422163 \\
    \bottomrule
\end{tabular}
\end{table}
}

所有测试均基于最新的Arch Linux操作系统,内核版本为4.19.12,搭载主频为3.6GHz的Intel Core i7-7700处理器,内存为8GB。运行测试期间关闭其他所有应用程序,每一组测试的结果均取10次运行后的平均值。

\section{结果对比}

为了保证公平,所有求解器尽可能地设置成相同的算法,并将算法运行时间$t_{\text{solve}}$定义为为的总运行时间减去冷启动所用时间。所有优化器均使用稀疏求解策略,本文的优化策略非线性部分为DL法,线性部分为增量舒尔补-正规Cholesky分解/增量舒尔补-正规PCG迭代法。Ceres-Solver的优化策略非线性部分也设置为DL法,由于Ceres-Solver的限制,线性部分只能设置为舒尔补-雅各比QR分解法。将所有优化算法的迭代次数设置为相同,通过运行时间$T_{\text{solve}}$、迭代次数Iter和优化结束时的重投影误差$Err_{\text{reproj}}$和相对位姿约束误差$Err_{\text{IMU}}$来对比效率和精度。

{
\linespread{1}
\begin{table}[htb!]
\zihao{5}
\caption{测试结果:运行时间和迭代次数}
\label{tab:time}
\centering
\begin{tabular}[b]{l|cccc}
    \toprule
    $T_{\text{solve}}$/$Iter.$ &      Cubic & Venice      &    Cathedral \\ \midrule
    基准Ceres:Schur-QR        & $69$ms/$5$ & $125$ms/$6$ & $3860$ms/$5$ \\
                               &  $100.0\%$ & $100.0\%$   &    $100.0\%$ \\ \midrule
    本文算法:Schur-Cholesky   & $17$ms/$5$ & $78$ms/$6$  & $2334$ms/$5$ \\
                               &   $24.6\%$ & $62.4\%$    &     $60.5\%$ \\ \midrule
    本文算法:Schur-I-PCG      & $18$ms/$5$ & $80$ms/$6$  & $2423$ms/$5$ \\
                               &   $26.1\%$ & $64.0\%$    &     $62.8\%$ \\
    \bottomrule
\end{tabular}
\end{table}
}

{
\linespread{1}
\begin{table}[htb!]
\zihao{5}
\caption{测试结果:收敛时的误差}
\label{tab:energy}
\centering
\begin{tabular}[b]{l|cccc}
    \toprule $Err_{\text{reproj}}$/
    $Err_{\text{reproj}}+Err_{\text{IMU}}$ &                 Cubic & Venice                   &               Cathedral \\
    (优化前误差)                         & ($2.514\times10^3$) & ($5.641\times10^{-1}$) & ($7.475\times10^{0}$) \\ \midrule
    基准Ceres:Schur-QR                    &     $5.153\times10^2$ & $2.996\times10^{-2}$     &    $4.153\times10^{-1}$ \\
    本文算法:Schur-Cholesky               &     $4.982\times10^2$ & $3.166\times10^{-2}$     &    $4.246\times10^{-1}$ \\
    本文算法:Schur-I-PCG                  &     $5.004\times10^2$ & $2.996\times10^{-2}$     &    $5.524\times10^{-1}$ \\
    \bottomrule
\end{tabular}
\end{table}
}

\section{本章小结}

本章通过在模拟VISLAM数据集Cubic和开源VSLAM数据集上测试对比了本文提出的增量式舒尔补集束调整优化算法和流行的开源Ceres-Solver的求解效率和精度。可以看到在各个数据集上,相比开源的Ceres-Solver,本文提出的基于增量舒尔补的集束调整方法均在保证一定精度的前提下将求解时间减少了$75.4\%$至$36.0\%$不等,显著提升了效率。

值得注意的是,本文提出的集束调整框架仍然具有较高的通用性和可扩展性:除了集束调整问题,也可以求解一般的非线性最小二乘问题;除了提供的Cholesky分解法和PCG方法,用户也可以自定义新的线性求解方法,以供不同场景使用。
