\section{基于因子图的增量舒尔补算法}\label{sec:ischur}

iSAM2算法\citep{kaess2008isam,kaess2012isam2}创新性地在SLAM问题中使用了贝叶斯推断树来编码集束调整过程中的信息矩阵分解过程,达到高效地更新平方根信息矩阵的目的,显著地提升了全局优化的性能。其算法不仅适用于SLAM中的集束调整问题,也适用于一般的非线性最小二乘问题。也正是为了保证通用性,iSAM2难以在状态之间的关联特点高度统一的SLAM问题中发挥更高的性能。

例如在图~\ref{fig:sparse_matrix}中,正规方程的三维点部分(左上角部分)高度稀疏,呈对角块状,在求解时应该优先考虑这一部分的分解。iSAM2算法需要依赖COLAMD\citep{davis2004algorithm}算法来被动地检测矩阵分解的顺序,一方面需要额外的计算时间,另一方面也不一定能得到比经验更好的结果。

舒尔补虽然可以大幅加速集束调整的求解,但是在标准的批量式集束调整算法中,每一轮迭代重新构建舒尔补方程的过程仍然需要耗费大量的时间。如前面提到的,SLAM问题具有很好的局部性,每一轮求解一般都只有一小部分较新的变量有所更新,如果采用增量式的方法,每一轮迭代时只重新计算这些变量在舒尔补方程中对应的部分,则可以进一步减少计算量。

\begin{figure}[htb!]
    \centering
    \includegraphics[scale=.7]{Pictures/factor_graph.png}
    \caption{因子图}
    \label{fig:factor_graph}
\end{figure}

如图~\ref{fig:factor_graph}是一个常见集束调整问题的因子图示例,其对应的正规方程具有如图~\ref{fig:normal_eq}所示的稀疏结构。按照式\eqref{eq:schur_complement}构建相机部分方程的舒尔补,依次迭代计算每一个三维点状态对应的舒尔补部分,并累加到图~\ref{fig:reduced_sys}中的高亮部分所示的舒尔补中。图~\ref{fig:schur_complement}展示了计算舒尔补的第一次迭代。

\begin{figure}[htb!]
    \centering
    \includegraphics[scale=1]{Pictures/normal_eq.png}
    \caption{正规方程:左上角块对角矩阵部分对应三维点状态$1$至$6$的信息矩阵$\mathrm{P}$,右下角部分为相机状态$1$至$5$对应的信息矩阵$\mathrm{C}$,左下角和右上角为矩阵$\mathrm{W}$和$\mathrm{W}^\top$;中间列向量从上至下依次对应三维点状态$1$至$6$和相机状态$1$至$5$的迭代步;右侧部分从上至下依次对应三维点状态$1$至$6$和相机状态$1$至$5$。}
    \label{fig:normal_eq}
\end{figure}

\begin{figure}[htb!]
    \centering
    \includegraphics[scale=1]{Pictures/reduced_sys.png}
    \caption{舒尔补:高亮部分为相机状态对应舒尔补方程,其余部分为原方程。}
    \label{fig:reduced_sys}
\end{figure}

\begin{figure}[htb!]
    \centering
    \includegraphics[width=\textwidth]{Pictures/schur_complement.png}
    \caption{计算舒尔补的一次迭代:从左上角开始依次迭代计算每一个三维点对应的舒尔补部分,并累加到右下角相机部分中。}
    \label{fig:schur_complement}
\end{figure}

\subsection{标记脏子图}

\begin{figure}[htb!]
    \centering
    \includegraphics[scale=.7]{Pictures/factor_graph_dirty.png}
    \caption{因子图待更新部分:相机状态$5$为脏变量;与其直接相连的因子为脏因子;相机状态$4$和三维点状态$6$为只有梯度发生了变化的脏变量。这些节点构成了一个脏子图。}
    \label{fig:factor_graph_dirty}
\end{figure}

增量更新舒尔补的方法关键在于尽可能地复用上一次迭代的结果,只针对性地计算舒尔补方程中变化较大的变量对应的部分。以因子图~\ref{fig:factor_graph}为例,假设只有相机状态$5$有较大的更新,则先将其标记为脏变量。在因子图中,只有与脏变量直接相连的因子需要标记为脏因子,如图~\ref{fig:factor_graph_dirty}所示。需要注意的是,尽管相机状态$5$和三维点状态$6$的值并未发生足够大的改变,但是由于与之相连的部分因子被标记为脏因子,因此也需要更新这些因子关于它们的雅各比矩阵。这些脏变量和脏因子共同组成了一个脏子图,对应地,图~\ref{fig:normal_eq_dirty}高亮标示了舒尔补方程中需要更新的脏块。算法~\ref{alg:mark_dirty}详细说明了如何通过一个简单的宽度优先搜索标记出因子图中的脏子图。

\begin{algorithm}
\caption{标记脏子图}
\begin{algorithmic}
    \Require 因子图$\check{\mathcal{G}}=\{\check{\mathcal{F}},\check{\Theta},\check{\mathcal{E}}\}$,
             迭代步长$\bm{\delta}$
    \Ensure 脏子图$\mathcal{G}=\{\mathcal{F},\Theta,\mathcal{E}\}$

    \State $\mathcal{F}\coloneqq\{\}$,
           $\Theta\coloneqq\{\}$,
           $\mathcal{E}\coloneqq\{\}$

    \ForAll{$\left\|\bm{\delta}_i\right\|_{\infty} \geq \varepsilon$}
        \State $\Theta\coloneqq\Theta\cup\{\bm{\theta}_i\}$
        \Comment{将变量$\bm{\theta}_i$标记为脏变量}

        \ForAll
        {$\{
                \bm{f}_j\in\check{\mathcal{F}} |
                \bm{\theta}_i\in\Theta_j
        \}$}
        \Comment{$\Theta_j$为与$\bm{f}_j$相连的变量集合}
            \State $\Theta\coloneqq\Theta\cup\Theta_j$

            \State $\mathcal{E} \coloneqq \mathcal{E} \cup \{e_{ij}\}$
            \Comment{记录变量$\bm{\theta}_i$与因子$\bm{f}_j$相连的边$e_{ij}$}

            \State $\mathcal{F} \coloneqq \mathcal{F} \cup \{\bm{f}_j\}$
            \Comment{将因子$\bm{f}_j$标记为脏因子}
        \EndFor
    \EndFor
\end{algorithmic}
\label{alg:mark_dirty}
\end{algorithm}


\begin{figure}[htb!]
    \centering
    \includegraphics[scale=1]{Pictures/normal_eq_dirty.png}
    \caption{图~\ref{fig:factor_graph_dirty}对应的舒尔补方程中的脏块:相机状态$4$、$5$和三维点状态$6$构成的线性子系统。}
    \label{fig:normal_eq_dirty}
\end{figure}

\subsection{增量更新舒尔补}

舒尔补方程中被标记为脏块的矩阵构成了一个线性子系统,对应因子图中的脏子图。如图~\ref{fig:schur_update}所展示的,在增量更新舒尔补的过程中,需要先从舒尔补方程中减去脏子图的贡献,随后重新线性化所有的脏因子,最后重新计算舒尔补。算法~\ref{alg:schur_update}详细描述了这个过程。需要注意的是,在计算舒尔补的过程中会重复用到$\mathrm{W}_i\mathrm{P}_{ii}^{-1}\mathrm{W}_i^\top$,$\mathrm{W}_i\mathrm{P}_{ii}^{-1}\bm{l}_i$,$\mathrm{J}_j^\top\mathrm{J}_j$,$\mathrm{J}_j^\top\bm{f}_j$,需要将这些中间项缓存下来,避免重复计算。

\input{Chapters/alg_schur_update.tex}

\begin{figure}[htb!]
    \centering
    \includegraphics[width=\textwidth]{Pictures/schur_update.png}
    \caption{更新舒尔补}
    \label{fig:schur_update}
\end{figure}

\subsection{状态增广}

除了状态变量发生较大变化的时候需要更新因子图和舒尔补方程,集束调整过程中,如果有新的状态或因子加入,也需要对因子图进行增广,并对和舒尔补方程进行相应更新。实际上,这个过程也可以认为是算法~\ref{alg:schur_update}的一个特例。如图~\ref{fig:factor_graph_aug},新加入的状态通过新的因子与原因子图中相关联的部分状态共同构成了一个子图,将这个子图标记为脏子图,即可使用算法~\ref{alg:schur_update}进行更新。算法~\ref{alg:factor_graph_aug}给出了详细的过程,为了提高效率,因子图的增广和舒尔补方程的更新会被同时执行。

\begin{figure}[htb!]
    \centering
    \includegraphics[scale=.7]{Pictures/factor_graph_aug.png}
    \caption{增广因子图}
    \label{fig:factor_graph_aug}
\end{figure}

\begin{figure}[htb!]
    \centering
    \includegraphics[scale=1]{Pictures/normal_eq_aug.png}
    \caption{增广正规方程}
    \label{fig:normal_eq_aug}
\end{figure}

\begin{figure}[htb!]
    \centering
    \includegraphics[scale=1]{Pictures/normal_eq_update.png}
    \caption{更新正规方程}
    \label{fig:normal_eq_update}
\end{figure}

\input{Chapters/alg_factor_graph_aug.tex}
