\begin{thanks}

乌飞兔走,暑往寒来,来浙大求学,不觉七载。虽不算长,但也经历了浙大三任校长,见证了百廿校庆。本硕七年,初见的喜悦,学业的挣扎,科研的起落,求职的焦虑,直到完成论文走上社会,也算是筚路蓝缕。毕业之际又逢连绵雨,感慨很多,颇有点“浮天水送无穷树,带雨云埋一半山”的意思。

学者必有师,回顾求学生涯,最须感谢的是在浙大的遇到的师长,尤其是我的导师章国锋教授和师兄李津羽博士。习坎示教,始见经纶,感谢章老师和李博循序渐进的指导,我能始见三维视觉这门学科的经纶,领悟求是二字的真谛。章老师为人为学的一丝不苟与兢兢业业,催人奋进;而李博同样学识渊博,于我来说既是兄长,也是老师。求学过程中能有这样的师长,幸甚至哉。其次要感谢的是视觉组和浙江商汤的同仁,感谢各位的支持和帮助,此地一别,祝今后的生活亦能保持赤诚之心,不负今日之志。

山一程,水一程,最想感谢的是父母和朋友。感谢父母对我生活和学业无条件的支持,以及对我所有选择的理解和尊重,儿在杭州一切都很顺利。也感谢一直以来的好友康俊、张健、思捷、信翰等人,篇幅有限,无法一一列举。与诸君相识多年,亦师亦友,互为陪伴。缱绻分兮,望来日再遇,仍旧可以桃李春风。

最后感谢我自己。求学之路漫长而艰辛,我犹如那莘莘征夫中的一员,每怀靡及,不过仍要感谢自己的热情和坚持。日后无论身居何位、境遇何如,切记教诲:求是、求真;无曰已是、无曰遂真。

\vspace*{\fill}
\begin{flushright}
    王儒 \\
    二〇一九年三月于求是园
\end{flushright}

\end{thanks}
