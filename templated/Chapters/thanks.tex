\begin{thanks}

  时光飞逝,不知不觉中,在浙江大学的研究生生涯就要结束了。回顾这两年多的时间,我感到收获很多。在此期间,我不仅积累了计算机视觉领域的许多知识、增长了大量实践经验,也逐渐完善了自己的世界观和价值观。带着这些收获,走出校园,步入社会,我感到信心满满。这两年多的一步步成长,离不开导师的指导和实验室伙伴们的帮助。

  首先,我要感谢我的导师鲍虎军教授和章国锋教授。鲍老师为我们创造了优越的研究环境,他对学生的指导细致入微、一针见血,在计算机图形学上的造诣让人印象深刻,非常感谢他在生活和学习上对我的指导和帮助。章老师一丝不苟的治学态度、勤奋的工作精神以及广博的知识让我深感敬佩,硕士期间,是章老师引领我步入了计算机视觉的殿堂,让我逐渐掌握了发现问题、解决问题的能力。此论文从选题到算法的设计与实现以及写作,都得到了章老师的悉心指导和斧正,感谢他在学习和生活上对我无私的帮助。

  其次,我要感谢视觉组的师兄弟姐妹们,感谢你们在学术上的无私分享和生活上对我的帮助,在这样一个向上、友善的实验室氛围里,无论是生活还是学习,我都感到轻松愉快。特别是15级的伙伴们:沈强、许\yan、蒋沁宏、孙汉林、吴香利,很荣幸在人生的这一阶段能和你们一起度过,感谢你们,和你们一起学习、生活的日子是充实而快乐的,这段岁月我将终身难忘。

  再次,感谢我的家人,感谢你们对我的支持和包容,感谢你们在物质上和精神上对我的无私支持,无论何时何地,你们都永远是我坚强的后盾。

  最后,感谢预审、盲审和答辩委员会的每一位老师,感谢你们在百忙之中抽出时间了解我的工作并对我的论文提出批评和修改意见,帮助我完善论文。


  \vspace*{\fill}
  \begin{flushright}
    李晨

    2017年1月于浙大紫金港校区CAD\&CG实验室
  \end{flushright}

\end{thanks}
