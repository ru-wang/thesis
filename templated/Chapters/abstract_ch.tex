\begin{abstract}
随着人工智能概念的兴起和增强现实、无人机、移动机器人、自动驾驶等行业的发展,工业界与学术界对高效率、高精度的鲁棒SLAM算法的需求越来越大,而SLAM应用中,状态估计方法的效率和精度极大地制约了SLAM算法的性能表现。目前主流的SLAM系统一般使用集束调整来进行非线性状态估计。一些系统使用了开源的通用非线性最小二乘求解器,为了适应不同类型的优化问题,这一类求解器通常采用批量式最小二乘算法,牺牲了效率;一些系统使用了增量式的集束调整算法以提升效率,但由于解法固定,只适用于特定类型的非线性目标函数和参数化方法,或与系统的耦合程度较高,通用性不佳。

本文基于以上分析,提出了面向视觉惯性SLAM的增量式通用集束调整框架,一方面使用了增量式集束调整算法以提升效率:1)本文使用了因子图来编码并辅助增量构建舒尔补的过程,在减少线性化和消元过程中的冗余计算的同时生成贝叶斯推断树;2)并针对基于增量式舒尔补的算法中常见的舒尔补矩阵秩亏的问题,提出了混合式LM-DL算法,在不破坏增量构建过程的前提下增强了线性解的数值稳定性;3)在变量回代求解部分,提出了基于贝叶斯推断树的PBT回代算法,以提高回代求解时的计算效率和变量一致性。

另一方面,该框架在保证一定效率和精度的基础上保持了很好的通用性,除了能求解视觉惯性SLAM的集束调整问题,也能求解一般的非线性最小二乘问题:1)该框架实现了视觉惯性SLAM中常用的IMU预积分目标函数和重投影误差目标函数,同时支持用户自定义任意类型的目标函数;2)并提供了SLAM中常用的旋转矩阵和四元数参数化方法,同时支持用户自定义任意类型的参数化方法;3)在线性部分,该框架提供了稀疏/稠密版本的Cholesky分解、QR分解和增量式预处理共轭梯度法等多种线性求解算法,并支持用户定制符合实际数值需求的线性求解算法。

本文提出的集束调整框架使用了对CPU缓存更为友好的块状稀疏矩阵来实现矩阵计算和存储,并在模拟数据和真实的集束调整数据集上进行了测试,通过实验验证了求解的效率和精度。

\keywords{SLAM,增量式舒尔补,贝叶斯推断树,集束调整,状态估计}
\end{abstract}
