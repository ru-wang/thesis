\begin{abstract}
    随着人工智能概念的兴起和AR、无人机、移动机器人、自动驾驶等行业的发展,工业界与学术界对高精度、高效率、高鲁棒性的三维感知的需求越来越大。在SLAM算法中,状态估计算法的速度和精度极大地制约了SLAM应用的性能。传统的批量式集束调整求解器为了保证通用性,牺牲了一定的效率,而现有的基于增量式集束调整的求解器虽然效率很高,但由于耦合程度过高,难以在其他SLAM系统中使用。本文基于以上分析,提出了适用于VISLAM的增量式集束调整通用求解求解框架,在保证效率和精度的基础上保持了很好的通用性。

    在非线性部分,本文提出的增量集束调整框架提供了基于IMU预积分的目标函数接口和重投影误差目标函数接口,以及多种变量参数化方法,用户也可以通过自定义实现所需的目标函数和参数化方法;在线性化部分,本文使用了增量舒尔补的方法大幅提升了线性化过程的效率,并针对基于增量舒尔补和Dog-Leg算法增量求解器中容易遇到的舒尔补矩阵秩亏的问题,提出了混合式LM-DL算法,在不影响舒尔补增量更新过程的前提下增强了线性求解的数值稳定性;在线性求解部分,本文提供了Cholesky分解和增量式预处理共轭梯度法算法实现,用户也可以提出了基于贝叶斯树的PBT回代算法,提高回代求解时的效率和变量一致性。

    整个集束调整框架完全基于块状稀疏矩阵实现,并在在模拟数据和SLAM++提供的集束调整数据集上进行了测试,通过实验验证了求解器的效率和精度。

    \keywords{SLAM,增量式舒尔补,贝叶斯树,集束调整,状态估计}
\end{abstract}
