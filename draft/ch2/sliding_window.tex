\section{滑动窗口优化}

离线的SfM算法对实时性的要求不高,故一般会使用全局集束优化来求解所有状态的最优值。但在SLAM算法中,频繁求解完整的集束优化将会严重降低系统的性能,特别是当地图规模增长到一定程度后,完整集束优化的时间复杂度往往以至少立方的速度增长,这对于AR、自动驾驶等对实时性要求很高的应用是不可接受的。

与SfM算法不同,基于SLAM的应用往往更关心系统最新的位姿状态和路标点状态,而对历史状态的需求较低。针对这一情况,PTAM\citep{klein2007parallel}最早提出了区分局部优化和全局优化的架构。即使用一个前台线程负责快速求解一个规模较小局部集束优化,以快速获取最新状态的粗略估计;另外一个后台线程以更低的频率对所有状态进行全局集束优化,对状态估计问题进一步优化。这种设计既保证了前台状态估计的时间上限,又保证了全局状态的一致性,如今已经成为了SLAM系统主流设计思路。

\subsection{条件状态估计}

\subsection{边缘状态估计}
