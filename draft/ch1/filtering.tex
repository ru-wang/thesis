\section{基于扩展卡尔曼滤波的状态估计}

在SLAM领域中,主流的状态估计算法有两类:基于扩展卡尔曼滤波(extended Kalman filter,以下简称EKF)的滤波法和基于非线性最小二乘(nonlinear least squares)的优化法。两类方法没有本质上的区别,都是使用最大化后验概率(maximum a-posteriori estimation)的思想求解非线性系统的最优状态分布。本节将介绍EKF的基本原理。

\subsection{扩展卡尔曼滤波}

EKF将状态估计的过程分为状态预测(predict)和状态更新(update)两个步骤,分别对应状态的转移模型(propagation model)和观测模型(observation model)。在状态预测阶段,EKF通过状态转移模型预测当前系统状态的先验分布,而在状态更新阶段,EKF通过当前时刻的观测模型和观测值对状态的分布进行修正,得到后验状态分布:
\begin{equation}
\begin{array}{rl}
    \text{转移模型} & \bm{x}_{k+1} = f(\bm{x}_k,\bm{u}_k,\bm{w}_k) \\
    \text{观测模型} & \bm{z}_{k+1} = h(\bm{x}_{k+1},\bm{v}_{k+1})
\end{array}
\end{equation}
其中$\bm{u}_k$为输入控制变量,$\bm{w}_k$和$\bm{v}_{k+1}$分别为独立的系统噪声和观测噪声,均符合零均值高斯分布:
\begin{equation}
\begin{aligned}
    p(\bm{w}_k)     &\sim \mathcal{N}(\bm{0},\mathrm{Q}) \\
    p(\bm{v}_{k+1}) &\sim \mathcal{N}(\bm{0},\mathrm{R})
\end{aligned}
\end{equation}

\subsubsection{状态预测}

对于一个离散时间的非线性系统,EKF假设已知$t_k$时刻系统的状态$\bm{x}_k$符合高斯分布:
\begin{equation}
    p(\bm{x}_k) \sim \mathcal{N}(\hat{\bm{x}}_{k|k},\mathrm{P}_{k|k})
\end{equation}
由于系统噪声$\bm{w}_k$未知,在$t_{k+1}$时刻可以通过转移模型预测当前系统状态的先验分布:
\begin{equation}
\begin{aligned}
    p(\bm{x}_{k+1}) &\sim \mathcal{N}(\hat{\bm{x}}_{k+1|k},\mathrm{P}_{k+1|k}) \\
    \hat{\bm{x}}_{k+1|k} &= f(\hat{\bm{x}}_{k|k},\bm{u}_k,\bm{0}) \\
    \mathrm{P}_{k+1|k}   &= \mathrm{A}_{k+1} \mathrm{P}_{k|k} \mathrm{A}_{k+1}^\top +
                            \mathrm{W}_{k+1}\mathrm{Q}_k\mathrm{W}_{k+1}^\top
\end{aligned}
\end{equation}
其中
\begin{equation}
\begin{aligned}
    \mathrm{A}_{k+1} &\equiv
        \frac{\partial f}
             {\partial\bm{x}}(\hat{\bm{x}}_{k|k},\bm{u}_k,\bm{0}) \\
    \mathrm{W}_{k+1} &\equiv
        \frac{\partial f}
             {\partial\bm{w}}(\hat{\bm{x}}_{k|k},\bm{u}_k,\bm{0})
\end{aligned}
\end{equation}

\subsubsection{状态更新}

假设在$t_{k+1}$时刻获得了对系统的观测$\bm{z}_{k+1}$而观测噪声$\bm{v}_{k+1}$未知,根据观测模型可以得到观测值的条件概率分布:
\begin{equation}
\begin{aligned}
    p(\bm{z}_{k+1}|\bm{x}_{k+1}) &\sim \mathcal{N}(\hat{\bm{z}}_{k+1}, \mathrm{S}_{k+1}) \\
    \hat{\bm{z}}_{k+1} &= h(\hat{\bm{x}}_{k+1|k},\bm{0}) \\
    \mathrm{S}_{k+1} &= \mathrm{V}_{k+1}\mathrm{R}_{k+1}\mathrm{V}_{k+1}^\top
\end{aligned}
\end{equation}
其中
\begin{equation}
\begin{aligned}
    \mathrm{H}_{k+1} &\equiv
        \frac{\partial h}
             {\partial\bm{x}}(\hat{\bm{x}}_{k+1|k},\bm{0}) \\
    \mathrm{V}_{k+1} &\equiv
        \frac{\partial h}
             {\partial\bm{v}}(\hat{\bm{x}}_{k+1|k},\bm{0})
\end{aligned}
\end{equation}

为了简洁表示,以下省略下标。由于后验概率$p(\bm{x}|\bm{z}) \propto p(\bm{x})p(\bm{z}|\bm{x})$,可以构建最大化后验概率问题:
\begin{equation}
\begin{aligned}
    \max_{\bm{x}}
        &\frac{1}{\sqrt{2\pi}\mathrm{P}^{\nicefrac{1}{2}}}
        \cdot \exp \left( -\tfrac{1}{2}
            \left\| \bm{x}-\hat{\bm{x}} \right\|
            _{\mathrm{P}^{-1}}^2
        \right) \\
    \cdot
        &\frac{1}{\sqrt{2\pi}\mathrm{S}^{\nicefrac{1}{2}}}
        \cdot \exp \left( -\tfrac{1}{2}
            \left\| \bm{z}-\hat{\bm{z}} \right\|
            _{\mathrm{S}^{-1}}^2
        \right)
\end{aligned}
\end{equation}
EKF假设状态先验$\hat{\bm{x}}$足够接近最优解$\bm{x}$,且$\bm{e}\equiv\bm{x}-\hat{\bm{x}}$。估最大化上式相当于最小化:
\begin{equation}
    \min_{\bm{x}} \tfrac{1}{2}
        \left\| \bm{e} \right\|_{{\mathrm{P}}^{-1}}^2 +
    \tfrac{1}{2}
        \left\| \bm{z}-h(\hat{\bm{x}},\bm{0}) \right\|_{\mathrm{S}^{-1}}^2
\end{equation}
\begin{equation}
\begin{aligned}
    \bm{e} &=
    \left( \mathrm{P} + \mathrm{H}^\top\mathrm{S}^{-1}\mathrm{H} \right)^{-1}
    \mathrm{H}^\top\mathrm{S}^{-1}(\bm{z}-h(\hat{\bm{x}},\bm{0}))
    \\
    &= \mathrm{P}\mathrm{H}^\top
    \left( \mathrm{H}\mathrm{P}\mathrm{H}^\top + \mathrm{S} \right)^{-1}
    (\bm{z}-h(\hat{\bm{x}},\bm{0}))
\end{aligned}
\end{equation}

记$t_{k+1}$时刻的卡尔曼增益(Kalman gain)为
\begin{equation}
    \mathrm{K}_{k+1} \equiv
        \mathrm{P}_{k+1|k} \mathrm{H}_{k+1}^\top
        \left(
            \mathrm{H}_{k+1} \mathrm{P}_{k+1|k} \mathrm{H}_{k+1}^\top +
            \mathrm{S}_{k+1}
        \right)^{-1}
\end{equation}
可以对状态进行更新:
\begin{equation}
    \hat{\bm{x}}_{k+1|k+1} =
        \hat{\bm{x}}_{k+1|k} + \mathrm{K}_{k+1}
        (\bm{z}_{k+1}-h(\hat{\bm{x}}_{k+1|k},\bm{0}))
\end{equation}
同时,要更新后验协方差矩阵:
\begin{equation}
    \mathrm{P}_{k+1|k+1} =
        \left( \mathrm{I} - \mathrm{K}_{k+1} \mathrm{H}_{k+1} \right)
        \mathrm{P}_{k+1|k}
        \left( \mathrm{I} - \mathrm{K}_{k+1} \mathrm{H}_{k+1} \right)^\top +
        \mathrm{K}_{k+1} \mathrm{S}_{k+1} \mathrm{K}_{k+1}^\top
\end{equation}
因此,在状态更新阶段,可以得到状态的后验概率分布:
\begin{equation}
    p(\bm{x}_{k+1}) \sim \mathcal{N}(\hat{\bm{x}}_{k+1|k+1},\mathrm{P}_{k+1|k+1})
\end{equation}
