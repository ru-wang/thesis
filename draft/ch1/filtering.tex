\section{基于扩展卡尔曼滤波的状态估计}

在SLAM领域中,主流的状态估计算法有两类:基于扩展卡尔曼滤波(extended Kalman filter,以下简称EKF)的滤波法和基于非线性最小二乘(nonlinear least squares)的优化法。两类方法没有本质上的区别,都是使用最大化后验概率(maximum a-posteriori estimation)的思想求解非线性系统的最优状态分布。本节将介绍滤波法的基本原理和方法,以及基于滤波法的MSCKF方法。

\subsection{扩展卡尔曼滤波}

EKF将状态估计的过程分为状态预测(predict)和状态更新(update)两个步骤,分别对应状态的转移模型(propagation model)和观测模型(observation model)。在状态预测阶段,EKF通过状态转移模型预测当前系统状态的先验分布,而在状态更新阶段,EKF通过当前时刻的观测模型和观测值对状态的分布进行修正,得到后验状态分布:
\begin{equation}
\begin{array}{rl}
    \text{转移模型} & \bm{x}_{k+1} = f(\bm{x}_k,\bm{u}_k,\bm{w}_k) \\
    \text{观测模型} & \bm{z}_{k+1} = h(\bm{x}_{k+1},\bm{v}_{k+1})
\end{array}
\end{equation}
其中$\bm{u}_k$为输入控制变量,$\bm{w}_k$和$\bm{v}_{k+1}$分别为独立的系统噪声和观测噪声,均符合零均值高斯分布:
\begin{equation}
\begin{aligned}
    \bm{w}_k     &\sim \mathcal{N}(\bm{0},\mathrm{Q}) \\
    \bm{v}_{k+1} &\sim \mathcal{N}(\bm{0},\mathrm{R})
\end{aligned}
\end{equation}

对于一个离散时间的非线性系统,EKF假设已知$t_k$时刻系统的状态$\bm{x}_k$符合高斯分布:
\begin{equation}
    \bm{x}_k \sim \mathcal{N}(\hat{\bm{x}}_{k|k},\mathrm{P}_{k|k})
\end{equation}
由于系统噪声$\bm{w}_k$未知,在$t_{k+1}$时刻可以通过转移模型得到当前系统状态的预测:
\begin{equation}
    \hat{\bm{x}}_{k+1|k} = f(\hat{\bm{x}}_{k|k},\bm{u}_k,\bm{0})
\end{equation}
假设在$t_{k+1}$时刻获得了对系统的观测$\bm{z}_{k+1}$而观测噪声$\bm{v}_{k+1}$未知,根据观测模型可以获取:
\begin{equation}
    \hat{\bm{z}}_{k+1|k} = h(\hat{\bm{x}}_{k+1|k},\bm{0})
\end{equation}
