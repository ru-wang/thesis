\section{增量式优化方法}

iSAM2算法\cite{kaess2008isam,kaess2012isam2}创新性地在SLAM问题中使用了贝叶斯树来编码集束优化过程中的信息矩阵分解过程,达到高效地更新平方根信息矩阵的目的,显著地提升了全局优化的性能。其算法不仅适用于SLAM中的集束优化问题,也适用于一般的非线性最小二乘问题。也正是为了保证通用性,iSAM2难以在状态之间的关联特征高度统一的SLAM问题中发挥更高的性能。

例如在图\ref{fig:sparse_matrix}中,正规方程的三维点部分(左上角部分)高度稀疏,呈对角块状,在求解时应该优先考虑这一部分的分解。iSAM2算法需要依赖COLAMD\citep{davis2004algorithm}算法来被动地检测矩阵分解的顺序,且由这类启发式算法得出的分解顺序未必是最符合问题实际的。
