\begin{abstract}
随着人工智能概念的兴起和AR、无人机、移动机器人、自动驾驶等行业的发展,工业界与学术界对高精度、高效率、高鲁棒性的三维感知的需求越来越大。在SLAM算法中,状态估计算法的速度和精度极大地制约了SLAM应用的性能。传统的批量式集束优化求解器为了保证通用性,牺牲了一定的效率,而现有的基于增量式集束优化的求解器虽然效率很高,但由于耦合程度过高,难以在其他SLAM系统中使用。本文基于以上分析,提出了适用于VISLAM的增量式集束优化通用求解器,提供了基于IMU预积分的目标函数接口和重投影误差目标函数接口,以及多种线性求解算法接口,在保证效率和精度的基础上保持了很好的通用性。

With the rise of artificial intelligence concepts and the development of industries such as AR, drones, mobile robots, and autonomous driving, there is a growing demand for high-precision, high-efficiency, and highly robust 3D sensing in industry and academia. In the SLAM algorithm, the speed and accuracy of the state estimation algorithm greatly limits the performance of the SLAM application. The traditional batch bundle adjustment solver sacrifices efficiency in order to ensure versatility. Existing solver based on incremental bundle adjustment is very efficient, but difficult to integrate into other SLAM systems due to the high degree of coupling. Based on the above analysis, this paper proposes a general incremental bundle adjustment solver for VISLAM, which provides built-in cost functions based on IMU preintegration and reprojection error, while retaining the interfaces for customized cost function types. Meanwile, we provide a variety types of linear solver algorithm, which ensure efficiency and versatility.
\end{abstract}
\addcontentsline{toc}{chapter}{摘要}
